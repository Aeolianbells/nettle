\documentclass[a4paper]{article}
\usepackage[utf8]{inputenc}
\usepackage{amsmath}
\usepackage{url}

\author{Niels Möller}
\title{Notes on ECC formulas}

\begin{document}

\maketitle

\section{Weierstrass curve}

Consider only the special case
\begin{equation*}
  y^2 = x^3 - 3x + b (mod p)     
\end{equation*}
See \url{http://www.hyperelliptic.org/EFD/g1p/auto-shortw.html}.

Affine formulas for duplication, $(x_2, y_2) = 2(x_1, y_1)$:
\begin{align*}
  t &=  (2y)^{-1} 3 (x_1^2 - 1) \\
  x_2 &= t^2 - 2 x_1 \\
  y_2 &= (x_1 - x_2) * t - y_1
\end{align*}
Affine formulas for addition, $(x_3, y_3) = (x_1, y_1) + (x_2,
y_2)$:
\begin{align}
  t &= (x_2 - x_1)^{-1} (y_2 - y_1) \\
  x_3 &= t^2 - x_1 - x_2 \\
  y_3 &= (x_1 - x_3) t - y_1
\end{align}

\section{Montgomery curve}

Consider the special case
\begin{equation*}
  y^2 = x^3 + b x^2 + x  
\end{equation*}
See \url{http://www.hyperelliptic.org/EFD/g1p/auto-montgom.html}.

Affine formulas for duplication, $(x_2, y_2) = 2(x_1, y_1)$:
\begin{align*}
  t &= (2 y_1)^{-1} (3 x_1^2 + 2b x_1 + 1) \\
  x_2 &= t^2 - b - 2 x_1 \\
  y_2 &= (3 x_1 + b) t - t^3 - y_1 \\
  &= (3 x_1 + b - t^2) t - y_1 \\
  &= (x_1 - x_2) t - y_1
\end{align*}
So the computation is very similar to the Weierstraß case, differing
only in the formula for $t$, and the $b$ term in $x_2$.

Affine formulas for addition, $(x_3, y_3) = (x_1, y_1) + (x_2,
y_2)$:
\begin{align*}
  t &= (x_2 - x_1)^{-1} (y_2 - y_1) \\
  x_3 &= t^2 - b - x_1 - x_2 \\
  y_3 &= (2 x_1 + x_2 + b) t - t^3 - y_1 \\
  &= (2 x_1 + x_2 + b - t^2) t - y_1 \\
  &= (x_1 - x_3) t - y_1
\end{align*}
Again, very similar to the Weierstraß formulas, with only an
additional $b$ term in the formula for $x_3$.

\section{Edwards curve}

For an Edwards curve, we consider the special case
\begin{equation*}
  x^2 + y^2 = 1 + d x^2 y^2
\end{equation*}
See \url{http://cr.yp.to/papers.html#newelliptic}.

Affine formulas for addition, $(x_3, y_3) = (x_1, y_1) + (x_2,
y_2)$:
\begin{align*}
  t &= d x_1 x_2 y_1 y_2 \\
  x_3 &= (1 + t)^{-1} (x_1 y_2 + y_1 x_2) \\
  y_3 &= (1 - t)^{-1} (y_1 y_2 - x_1 x_2)
\end{align*}
With homogeneous coordinates $(X_1, Y_1, Z_1)$ etc., D.~J.~Bernstein
suggests the formulas
\begin{align*}
  A &= Z_1 Z_2 \\
  B &= A^2 \\
  C &= X_1 X_2 \\
  D &= Y_1 Y_2 \\
  E &= d C D \\
  F &= B - E \\
  G &= B + E \\
  X_3 &= A F [(X_1 + Y_1)(X_2 + Y_2) - C - D] \\
  Y_3 &= A G (D - C) \\
  Z_3 &= F G
\end{align*}
This works also for doubling, but a more efficient variant is
\begin{align*}
  B &= (X_1 + Y_1)^2 \\
  C &= X_1^2 \\
  D &= Y_1^2 \\
  E &= C + D \\
  H &= Z_1^2 \\
  J &= E - 2H \\
  X_3 &= (B - E) J \\
  Y_3 &= E (C - D) \\
  Z_3 &= E J
\end{align*}

\section{Curve25519}

Curve25519 is defined as the Montgomery curve
\begin{equation*}
  y^2 = x^3 + b x^2 + x \pmod p
\end{equation*}
with $b = 486662$ and $p = 2^{255} -19$. It is equivalent to the
Edwards curve
\begin{equation*}
  u^2 + v^2 = 1 + d u^2 v^2 \pmod p
\end{equation*}
with $d = (121665/121666) \bmod p$. The equivalence is given by
mapping $P = (x,y)$ to $P' = (u, v)$, as follows.
\begin{itemize}
\item $P = \infty$ corresponds to $P' = (0, 1)$
\item $P = (0, 0)$ corresponds to $P' = (0, -1)$
\item Otherwise, for all other points on the curve. First note that $x
  \neq -1$ (since then the right hand side is a not a quadratic
  residue), and that $y \neq 0$ (since $y = 0$ and $x \neq 0$ implies
  that $x^2 + bx + 1 = 0$, or $(x + b/2)^2 = (b/2)^2 - 1$, which also
  isn't a quadratic residue). The correspondence is then given by
  \begin{align*}
    u &= \sqrt{b+2} \, x / y \\
    v &= (x-1) / (x+1)
  \end{align*}
\end{itemize}

The inverse transformation is
\begin{align*}
  x &= (1+v) / (1-v) \\
  y &= \sqrt{b+2} x / u 
\end{align*}
If the Edwards coordinates are represented using homogeneous
coordinates, $u = U/W$ and $v = V/W$, then
\begin{align*}
  x &= \frac{W+V}{W-V} \\
  y &= \sqrt{b} \frac{(W+V) W}{(W-V) U} 
\end{align*}
so we need to invert the value $(W-V) U$.
\end{document}

%%% Local Variables: 
%%% mode: latex
%%% TeX-master: t
%%% End: 
